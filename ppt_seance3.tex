\documentclass{beamer}
\usepackage{etex}

\usepackage[utf8]{inputenc}
\usepackage[T1]{fontenc}

\usepackage[french]{babel}

\usepackage{amsmath} 
\usepackage{graphicx}
%\usepackage{wrapfig}
\usepackage{amsmath}
\usepackage{array}
\usepackage{amsfonts}
%\usepackage{mathtools}
\usepackage{amssymb}
\usepackage{bbm}
\usepackage{pdfpages} %inclusion de fichier pdf!
%\usepackage[left,pagewise]{lineno}  
%\usepackage{float}   %pour forcer les figuester ou elles sont !! 
\usepackage{multibib} %bibliomultiple  cf doc "multibib" sur le net
\usepackage[normalem]{ulem} %barrer un mot
\usepackage{fancyvrb}
\usepackage{listings, multicol, xcolor} % Listings
\usepackage{lmodern}
\usepackage{url, hyperref} % Urls

%pour colorer du \verb nom \colorverb
\usepackage{xcolor}



% \meta command, a la doc
\newcommand{\meta}[1]{\ensuremath\langle\itshape#1\ensuremath\rangle}

\lstset{
upquote=false,
columns=flexible,
basicstyle=\ttfamily\scriptsize,
language={[LaTeX]TeX},
identifiersstyle=\color{green},
emphstyle=\color{blue},
keywordstyle=\color{blue},
directivestyle=\color{blue},
commentstyle=\color{gray},
    inputencoding=utf8,
literate={eacute}{\'e}1,
    {eagrave}{\`e}1,
    {aagrave}{\`a}1,
    escapechar={£},
    moretexcs={meta},
morekeywords={
  part,chapter,subsection,subsubsection,
  frontmatter,mainmatter,backmatter,
  tableofcontents,listoffigures,listoftables,titlepage,
  includegraphics,includepdf,
  dddot,ddddot,
  frametitle,framesubtitle,
  pause,only,uncover,
  usetheme,usecolortheme,
  institute,maketitle,
  usetikzlibrary,
  node,path,
  commande
}
}

\usetheme{CambridgeUS}

\usepackage{tikz}
\usetikzlibrary{chains}
\usetikzlibrary{calc, decorations.pathmorphing, fadings, shadings, arrows, decorations.pathreplacing, shapes} 
 %usepackage ect...
\usepackage{pdfpages}

\title [Séminaire \LaTeX, séance 3]{Séminaire \LaTeX, séance 3: utilisation avancée}
\date{jeudi 28 février 2013}
\author[Folschette, Jubien, Tanguy]{Maxime \textsc{Folschette\up{1} } \and Anthony \textsc{Jubien\up{2}} \and Julien \textsc{Tanguy\up{3}} \\ 
{\scriptsize  \up{1} IRCCyN équipe MeForBio \\
\up{2} IRCCyN équipe Robotique et ONERA Toulouse \\
\up{3} IRCCyN équipe Systèmes Temps Réels \\
maxime.folschette, anthony.jubien, julien.tanguy @irccyn.ec-nantes.fr} }
\institute[AED]{Association des Étudiants en Doctorat de l'ECN (AED)  \\ 
Document sous licence Creative Commons BY 3.0 FR \\
http://creativecommons.org/licenses/by/3.0/fr/}


\setbeamertemplate{navigation symbols}{}

\AtBeginPart{%
    \frame{\partpage}

    \begin{frame}
        \frametitle{Plan}
        \tableofcontents
    \end{frame}
}
 %author ect...

%%Macros exemple
\newcommand{\ltsname}{Labeled Transition System}
\newcommand{\abs}[1]{\ensuremath\left|#1\right|}
\newcommand{\lts}[1][]{\ensuremath\left(Q^{#1}, q_0^{#1}, A_{#1}, \rightarrow_{#1}\right)}
\renewcommand{\vec}[1]{\overrightarrow{#1}}

\begin{document}

%%%%%%%%% SLIDE %%%%%%%%%%%%%%%%%%

\begin{frame}
    \titlepage
\end{frame}

%%%%%%%%% SLIDE %%%%%%%%%%%%%%%%%%

\begin{frame}{Points abordés durant la séance 3:}
    \begin{itemize}
            \item bibliographie, 
            \item commandes avancées, 
            \item inclusion de figures à l'aide de différents outils, 
            \item création d'un diaporama à l'aide de la classe Beamer. 
        \end{itemize}
\end{frame}

%%%%%%%%%%%%%%%%%%%%%%%%%%%%%%
%%%%%%%%%%% PART%% %%%%%%%%%%%%%%
%%%%%%%%%%%%%%%%%%%%%%%%%%%%%%

\part{Bibliographie}

%%%%%%%%%%

\section{BibTeX}

%%%%%%%%%% SLIDE %%%%%%%%%%%%%%%%%%

\begin{frame}[fragile]
\frametitle{Présentation de BibTeX}
BibTeX est un outil de gestion de bibliographie
	

La \emph{base de données} bibliographique est placée dans un fichier extérieur (.bib)
On le place dans le document par les commandes:
\begin{lstlisting}
\bibliographystyle{plain}
\bibliography{nom-biblio}
\end{lstlisting}
On y fait référence par la commande \lstinline?\cite{...}?  \cite{latexcompanion}

Il est possible d'inclure plusieurs fichiers .bib:
\lstinline?\bibliography{biblio1,biblio2}?
\end{frame}

%%%%%%%%%%

\section{Exemple}

%%%%%%%%% SLIDE %%%%%%%%%%%%%%%%%%

\begin{frame}[fragile]
\frametitle{Exercice}

Créer un nouveau fichier .tex nommé biblio.tex. 

\begin{lstlisting}
@article{greenwade93,
    author  = "Inconnu",
    title   = "Titre",
    year    = "1993",
    journal = "Nom du journal",
    volume  = "14",
    number  = "3",
    pages   = "342--351"
}
\end{lstlisting}

Et y faire référence dans votre document principal:

\begin{lstlisting}
....
\cite{greenwade93}
....
\bibliographystyle{plain} %ou style alpha
\bibliography{biblio}
\end{lstlisting}

\end{frame}


%%%%%%%%%%

\section{JabRef}

%%%%%%%%%% SLIDE %%%%%%%%%%%%%%%%%%

\begin{frame}
\frametitle{Outils de gestion de bibliographie}

La plupart des bases de données bibliographiques permettent d'exporter une entrée en BibTeX (Google Scholar inclu: Préférences Scholar,  Gestionnaire des bibliographies,  Afficher les liens permettant d'importer des citations dans BibTeX).

Utiliser un outil de gestion de bibliographie est nécessaire:
\begin{itemize}
\item JabRef,
\item Mendeley,
\item Zotero
\end{itemize}
\end{frame}

%%%%%%%%% SLIDE %%%%%%%%%%%%%%%%%%

\begin{frame}{Jabref (mutli-plateforme)}

\vspace*{-0.5cm}
\begin{figure} %debut de l'environnement figure
\centering %figure centree
\includegraphics[width=9cm]{img/jabref} % image de X cm de large
%\caption{Titre de la figure} %titre de la figure
%\label{titre_fig} %label de la figure (ex voir figure W)
\end{figure} %fin de l'environnement figure

\vspace*{-0.5cm}
{\footnotesize Téléchargement: http://jabref.sourceforge.net/ }

\end{frame}


%%%%%%%%%%%%%%%%%%%%%%%%%%%%%%
%%%%%%%%%%% PART%% %%%%%%%%%%%%%%
%%%%%%%%%%%%%%%%%%%%%%%%%%%%%%


\part{Commandes avancées}

\section{Commandes personnalisées}

\begin{frame}[fragile]
	\frametitle{Créer ses propres commandes}
	Pourquoi?
	\begin{itemize}
		\item Réutilisation
		\item Simplification
	\end{itemize}
	
	Définition

	\begin{lstlisting}
\newcommand{\ltsname}{Labeled Transition System}
\newcommand{\abs}[1]{\left|#1\right|}
\newcommand{\lts}[1][]{\left(Q^{#1},q_0^{#1},A_{#1},\rightarrow_{#1}\right)}
	\end{lstlisting}
	Utilisation
		\begin{itemize}
	\item \lstinline?\ltsname? $\Rightarrow$ \ltsname
	\item \lstinline?\abs{\pi}? $\Rightarrow$ $\abs{\pi}$
	\item \lstinline?\lts? $\Rightarrow$ $\lts$\\
		\lstinline?\lts[n]? $\Rightarrow$ $\lts[n]$
	\end{itemize}
	Restrictions
	\begin{itemize}
		\item Pas de chiffres
		\item Pas de @
	\end{itemize}
\end{frame}

\begin{frame}[fragile]
	\frametitle{Redéfinir des commandes}

	\begin{lstlisting}
\renewcommand{\vec}[1]{\overrightarrow{#1}}
	\end{lstlisting}
	Utilisation
		\begin{itemize}
	\item \lstinline?\vec{AB}? $\Rightarrow$ $\vec{AB}$
	\end{itemize}

\end{frame}

\section{Comprendre la compilation}

\begin{frame}
	\frametitle{Fichiers auxiliaires}
	\begin{description}
	\item[log] fichier où \LaTeX{} écrit tout un tas d'informations sur la dernière compilation
	\item[aux] fichier auxiliaire: stocke les références, citations, numéros de page, etc.
	\item[toc] fichier contenant la table des matières
	\item[lof] fichier contenant la liste des figures
	\item[lot] fichier contenant la liste des tables
	\item[bbl] fichier contenant la bibliographie
	\end{description}
\end{frame}

\setbeamercovered{transparent}
\begin{frame}
\frametitle{Cycle de compilation}
\begin{center}
    \begin{tikzpicture}[every node/.style={shape=rectangle, shape aspect=1.61, rounded corners},
                        every path/.style={thick, >=triangle 60},
                        file/.style={minimum height=0.8cm, minimum width=1.29, fill=blue!20, draw=blue},
                        bin/.style={minimum width=2.42cm, minimum height=1.5cm, fill=red!20, draw=red}]
                        \only<1, 3-4>{\node[bin] (bin) at (0,0) {\LaTeX{}};}
                        \only<2>{\node[bin] (bin) at (0,0) {Bib\TeX{}};}
                        \onslide<1, 3-4>{\node[file] (tex) at (-3, 1) {\texttt{.tex}};}
                        \onslide<2>{\node[file] (bib) at (-3, 0) {\texttt{.bib}};}
                        \onslide<3-4>{\node[file] (bst) at (-3, -1) {\texttt{.bst}};}

                        \onslide<3-4>{\node[file] (pdf) at (3,0.75) {\texttt{.pdf}};}
                        \onslide<1, 3-4>{\node[file] (log) at (3, -0.75) {\texttt{.log}};}

                        \onslide<2-4>{\node[file] (aux) at (-1.5, 2) {\texttt{.aux}};}
                        \onslide<3-4>{\node[file] (bbl) at (0, 2) {\texttt{.bbl}};}
                        \onslide<2>{\node[file] (blg) at (1.5, 2) {\texttt{.blg}};}

                        \onslide<4>{\node[file] (toc) at (-1.5, -2) {\texttt{.toc}};}
                        \onslide<4>{\node[file] (lof) at (0, -2) {\texttt{.lof}};}
                        \onslide<4>{\node[file] (lot) at (1.5, -2) {\texttt{.lot}};}

                        \only<1>{
                            \draw[->] (tex) -- (bin);
                            \draw[->] (bin) -- (log);
                            \draw[->] (bin) -- (pdf);
                            \draw[->] (bin) -- (aux);
                        }
                        \only<2>{
                            \draw[->] (bib) -- (bin);
                            \draw[->] (aux) -- (bin);
                            \draw[->] (bin) -- (blg);
                            \draw[->] (bin) -- (bbl);
                        }
                        \only<3>{
                            \draw[->] (tex) -- (bin);
                            \draw[->] (aux) -- (bin);
                            \draw[->] (bst) -- (bin);
                            \draw[->] (bbl) -- (bin);
                            \draw[->] (bin) -- (log);
                            \draw[->] (bin) -- (pdf);
                            \draw[->] (bin) -- (toc);
                            \draw[->] (bin) -- (lof);
                            \draw[->] (bin) -- (lot);
                        }
                        \only<4>{
                            \draw[->] (tex) -- (bin);
                            \draw[->] (aux) -- (bin);
                            \draw[->] (bst) -- (bin);
                            \draw[->] (bbl) -- (bin);
                            \draw[->] (bin) -- (log);
                            \draw[->] (bin) -- (pdf);
                            \draw[<-] (bin) -- (toc);
                            \draw[<-] (bin) -- (lof);
                            \draw[<-] (bin) -- (lot);
                        }
    \end{tikzpicture}
\end{center}
\end{frame}

\setbeamercovered{invisible}
%%%%%%%%%%%%%%%%%%%%%%%%%%%%%%
%%%%%%%%%%% SECTION %%%%%%%%%%%%%%
%%%%%%%%%%%%%%%%%%%%%%%%%%%%%%

\section{Autres éditeurs \LaTeX}


%%%%%%%%% SLIDE %%%%%%%%%%%%%%%%%%

\begin{frame}{Texniccenter}

\begin{figure} %debut de l'environnement figure
\centering %figure centree
\includegraphics[width=9cm]{img/Texniccenter} % image de X cm de large
%\caption{Titre de la figure} %titre de la figure
%\label{titre_fig} %label de la figure (ex voir figure W)
\end{figure} %fin de l'environnement figure

{\footnotesize Téléchargement: http://www.texniccenter.org/ }

\end{frame}


%%%%%%%%% SLIDE %%%%%%%%%%%%%%%%%%

\begin{frame}{LyX}

\begin{figure} %debut de l'environnement figure
\centering %figure centree
\includegraphics[width=8cm]{img/LyXScreen_Linux_en} % image de X cm de large
%\caption{Titre de la figure} %titre de la figure
%\label{titre_fig} %label de la figure (ex voir figure W)
\end{figure} %fin de l'environnement figure

{\footnotesize Téléchargement: http:// 	www.lyx.org/ }

\end{frame}


%%%%%%%%% SLIDE %%%%%%%%%%%%%%%%%%

\begin{frame}{Texmaker}

\begin{figure} %debut de l'environnement figure
\centering %figure centree
\includegraphics[width=9cm]{img/TexmakerView} % image de X cm de large
%\caption{Titre de la figure} %titre de la figure
%\label{titre_fig} %label de la figure (ex voir figure W)
\end{figure} %fin de l'environnement figure

{\footnotesize Téléchargement: http://www.xm1math.net/texmaker/ }

\end{frame}


%%%%%%%%%%%%%%%%%%%%%%%%%%%%%%
%%%%%%%%%%% PART%% %%%%%%%%%%%%%%
%%%%%%%%%%%%%%%%%%%%%%%%%%%%%%

\part{Inclusion de figures à l'aide de différents outils}

%TikZ
% Présentation de TikZ

\section{Présentation de PGF/TikZ}

\begin{frame}
  \frametitle{PGF/TikZ : du dessin vectoriel en \LaTeX}

Qu'est-ce que PGF/TikZ ?
\begin{itemize}
  \item PGF est un langage complet et compliqué de dessin vectoriel,
  \item TikZ est une surcouche plus simple pour utiliser PGF.
\end{itemize}

\bigskip
Ils permettent de dessiner des figures facilement. Beaucoup d'avantages :
\begin{itemize}
  \item les figures sont intégrés au document \LaTeX{} (pas de fichier externe),
  \item dessin vectoriel : toujours lisse, quel que soit le niveau de zoom,
  \item très riche, beaucoup d'exemples disponibles faciles à reprendre.
\end{itemize}

\bigskip
Inconvénients :
\begin{itemize}
  \item parfois difficile à prendre en main,
  \item peut alourdir la compilation et le fichier final,
  \item ne permet pas de tout faire (mais presque).
\end{itemize}
\end{frame}



\section{Quelques exemples avec TikZ}

\input{ppt_seance3_tikz_ex}

\begin{frame}
  \begin{figure}
    \centering
    \tikzexa
    \caption{\footnotesize Modèle d'architecture --- TEXample.net \cite{tikzandpgfexamples}}
  \end{figure}
\end{frame}

\begin{frame}
  \begin{figure}
    \centering
    \tikzexc
    \caption{\footnotesize Graphe simple --- TEXample.net \cite{tikzandpgfexamples}}
  \end{figure}
\end{frame}

\begin{frame}
  \begin{figure}
    \centering
    \scalebox{0.7}{\tikzexd}
    \caption{\footnotesize Incidence oblique --- TEXample.net \cite{tikzandpgfexamples}}
  \end{figure}
\end{frame}

\begin{frame}
  \begin{figure}
    \centering
    \scalebox{0.7}{\tikzexb}
    \caption{\footnotesize Microscope électronique à transmission --- TEXample.net \cite{tikzandpgfexamples}}
  \end{figure}
\end{frame}



\section{Utilisation de TikZ}

\begin{frame}[fragile]
  \frametitle{Préambule et création d'une figure}

TikZ doit être chargé dans le préambule :
\lstinline?\usepackage{tikz}?

\medskip
On peut aussi charger des bibliothèques propres à TikZ dans le préambule avec :
\lstinline?\usetikzlibrary{bibliotheques}?,
ce qui permet d'utiliser :
\begin{itemize}
  \item de nouvelles formes de pointes de flèches (\lstinline?arrows?),
  \item des dégradés (\lstinline?shadings?),
  \item des styles de lignes (\lstinline?decorations.pathmorphing?), ...
\end{itemize}

\bigskip
Dans le document, on définit une image TikZ à l'aide de l'environnement \lstinline?tikzpicture?,
souvent inclus dans une \lstinline?figure? :

\begin{lstlisting}
\begin{figure}
  \begin{tikzpicture}
    ...
    ...    % Contenu de l'image
    ...
  \end{tikzpicture}
  \caption{...}
\end{figure}
\end{lstlisting}
\end{frame}




\def \tikzexnodes
{
\node[circle, fill=yellow, draw]
  (rond) {1};
\node[ellipse, fill=red!50]
  (ellipse) [right of=1, node distance=3cm] {Une ellipse};
\node[diamond, fill=blue!50, draw=blue, thick]
  (diamantvide) [left of=1, node distance=2cm] {};
}

\def \tikzexedges
{
\path[->] (rond) edge (ellipse);
\path[o->, bend right] (rond) edge (diamantvide);
\path[o->>, bend right] (diamantvide) edge
  node[below, fill=green!30] (retour) {retour}
  (rond);
\path[-, bend right] (retour.east) edge (rond.south);
}

\begin{frame}[fragile,t]
  \frametitle{Exemple : un graphe simple}

\begin{figure}
  \begin{tikzpicture}
    \tikzexnodes
    \tikzexedges
  \end{tikzpicture}
\end{figure}

Une figure TikZ est constituée d'éléments définis à l'aide de commandes :

\begin{lstlisting}
  \commande[param£\`e£tres] ... suite de la commande ... ;
\end{lstlisting}

Par exemple, un graphe est composé de nœuds et d'arcs entre ces nœuds.
Tous sont définis à l'aide de commandes TikZ.

\medskip
On commence par définir une figure et une image TikZ :
\begin{lstlisting}
\begin{figure}
  \begin{tikzpicture}
    ...    % La figure ici
  \end{tikzpicture}
\end{figure}
\end{lstlisting}
\end{frame}

\begin{frame}[fragile, t]
  \frametitle{Exemple : un graphe simple}

\begin{figure}
  \begin{tikzpicture}
    \tikzexnodes
  \end{tikzpicture}
\end{figure}

On définit un nœud avec la commande \lstinline?\node? et on peut spécifier :
\medskip
\begin{itemize}
  \item le nom interne \lstinline?(nom)? et l'étiquette visible \lstinline?{etiquette}?
  \item la forme \quad \lstinline?circle?, \lstinline?ellipse?, \lstinline?square?, \lstinline?diamond?
  \item le type de ligne \quad \lstinline?draw? : visible, \lstinline?draw=blue? : en bleu, \lstinline?thick? : en gras
  \item la couleur de fond \quad \lstinline?fill=yellow? : fond jaune, \lstinline?shading? : dégradé
  \item la position \quad \lstinline?left/right/above/below of=n? : position par rapport à \lstinline?n?
\end{itemize}

\begin{lstlisting}
\node[circle, fill=yellow, draw]
  (rond) {1};
\node[ellipse, fill=red!50]
  (ellipse) [right of=1, node distance=3cm] {Une ellipse};
\node[diamond, fill=blue!50, draw=blue, thick]
  (diamantvide) [left of=1, node distance=2cm] {};
\end{lstlisting}
\end{frame}

\begin{frame}[fragile, t]
  \frametitle{Exemple : un graphe simple}

\begin{figure}
  \begin{tikzpicture}
    \tikzexnodes
    \tikzexedges
  \end{tikzpicture}
\end{figure}

On définit ensuite un arc entre deux nœuds avec la commande

\begin{lstlisting}
    \path[£\meta{options}£] (£\meta{origine}£) edge (£\meta{cible}£);
\end{lstlisting}

On peut définir le type de flèche (\lstinline?->?, \lstinline?o->?, \lstinline?-?),
la courbure (\lstinline?bend right?),
etc.

\begin{lstlisting}
\path[->] (rond) edge (ellipse);
\path[o->, bend right] (rond) edge (diamantvide);
\end{lstlisting}

On peut placer un nouveau nœud sur un arc avec le mot-clef \lstinline?node?.

\begin{lstlisting}
\path[-, bend right] (retour.east) edge (rond.south);
\path[o->>, bend right] (diamantvide) edge
  node[below, fill=green!30] (retour) {retour} (rond);
\end{lstlisting}
\end{frame}



\section{Conclusion sur TikZ}

\begin{frame}
  \frametitle{Réutiliser au maximum}

Pour produire de belles figures TikZ, le mieux est de chercher des exemples et de les modifier.

\begin{center}
Pour cela : \Huge Internet !
\end{center}

On pourra notamment se servir des exemples de TEXample.net \cite{tikzandpgfexamples}.

\bigskip
De plus, il est possible de :
\begin{itemize}
  \item définir des thèmes pour des figures semblables,
  \item d'utiliser des bibliothèques pour des diagrammes répandus (UML, schémas électriques...).
\end{itemize}
\end{frame}


% Beamer
\input{ppt_seance3_beamer}

%%% Bibliographie %%%

\begin{frame}
  \frametitle{Bibliographie}

\nocite{*}
\bibliographystyle{abbrv-fr}
\bibliography{latex}
\end{frame}


%%%%%%%%%%%%%%%%%%%%%%%%%%%%%%%%%%%%%%%%%%%%%%%%%%%%%%%%%%%%%%%%%%%%%%%%%%%%%%%%%%
%%%%%%%%%%%%%%%%%%%%%%%% FIN DU DOCUMENT %%%%%%%%%%%%%%%%%%%%%%%%%%%%%%%%%%%%%%%%%%%%%%%%
%%%%%%%%%%%%%%%%%%% ( NON PRISE EN COMPTE DE LA SUITE ) %%%%%%%%%%%%%%%%%%%%%%%%%%%%%%%%%%%%%%%%%%%
\end{document}
%%%%%%%%%%%%%%%%%%%%%%%%%%%%%%%%%%%%%%%%%%%%%%%%%%%%%%%%%%%%%%%%%%%%%%%%%%%%%%%%%%
%%%%%%%%%%%%%%%%%%%%%%%%%%%%%%%%%%%%%%%%%%%%%%%%%%%%%%%%%%%%%%%%%%%%%%%%%%%%%%%%%%
