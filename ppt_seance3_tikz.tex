% Présentation de TikZ

\section{Présentation de PGF/TikZ}

\begin{frame}
  \frametitle{PGF/TikZ : du dessin vectoriel en \LaTeX}

Qu'est-ce que PGF/TikZ ?
\begin{itemize}
  \item PGF est un langage complet et compliqué de dessin vectoriel,
  \item TikZ est une surcouche plus simple pour utiliser PGF.
\end{itemize}

\bigskip
Ils permettent de dessiner des figures facilement. Beaucoup d'avantages :
\begin{itemize}
  \item les figures sont intégrés au document \LaTeX{} (pas de fichier externe),
  \item dessin vectoriel : toujours lisse, quel que soit le niveau de zoom,
  \item très riche, beaucoup d'exemples disponibles faciles à reprendre.
\end{itemize}

\bigskip
Inconvénients :
\begin{itemize}
  \item parfois difficile à prendre en main,
  \item peut alourdir la compilation et le fichier final,
  \item ne permet pas de tout faire (mais presque).
\end{itemize}
\end{frame}



\section{Quelques exemples avec TikZ}

\input{ppt_seance3_tikz_ex}

\begin{frame}
  \begin{figure}
    \centering
    \tikzexa
    \caption{\footnotesize Modèle d'architecture --- TEXample.net \cite{tikzandpgfexamples}}
  \end{figure}
\end{frame}

\begin{frame}
  \begin{figure}
    \centering
    \tikzexc
    \caption{\footnotesize Graphe simple --- TEXample.net \cite{tikzandpgfexamples}}
  \end{figure}
\end{frame}

\begin{frame}
  \begin{figure}
    \centering
    \scalebox{0.7}{\tikzexd}
    \caption{\footnotesize Incidence oblique --- TEXample.net \cite{tikzandpgfexamples}}
  \end{figure}
\end{frame}

\begin{frame}
  \begin{figure}
    \centering
    \scalebox{0.7}{\tikzexb}
    \caption{\footnotesize Microscope électronique à transmission --- TEXample.net \cite{tikzandpgfexamples}}
  \end{figure}
\end{frame}



\section{Utilisation de TikZ}

\begin{frame}[fragile]
  \frametitle{Préambule et création d'une figure}

TikZ doit être chargé dans le préambule :
\lstinline?\usepackage{tikz}?

\medskip
On peut aussi charger des bibliothèques propres à TikZ dans le préambule avec :
\lstinline?\usetikzlibrary{bibliotheques}?,
ce qui permet d'utiliser :
\begin{itemize}
  \item de nouvelles formes de pointes de flèches (\lstinline?arrows?),
  \item des dégradés (\lstinline?shadings?),
  \item des styles de lignes (\lstinline?decorations.pathmorphing?), ...
\end{itemize}

\bigskip
Dans le document, on définit une image TikZ à l'aide de l'environnement \lstinline?tikzpicture?,
souvent inclus dans une \lstinline?figure? :

\begin{lstlisting}
\begin{figure}
  \begin{tikzpicture}
    ...
    ...    % Contenu de l'image
    ...
  \end{tikzpicture}
  \caption{...}
\end{figure}
\end{lstlisting}
\end{frame}




\def \tikzexnodes
{
\node[circle, fill=yellow, draw]
  (rond) {1};
\node[ellipse, fill=red!50]
  (ellipse) [right of=1, node distance=3cm] {Une ellipse};
\node[diamond, fill=blue!50, draw=blue, thick]
  (diamantvide) [left of=1, node distance=2cm] {};
}

\def \tikzexedges
{
\path[->] (rond) edge (ellipse);
\path[o->, bend right] (rond) edge (diamantvide);
\path[o->>, bend right] (diamantvide) edge
  node[below, fill=green!30] (retour) {retour}
  (rond);
\path[-, bend right] (retour.east) edge (rond.south);
}

\begin{frame}[fragile,t]
  \frametitle{Exemple : un graphe simple}

\begin{figure}
  \begin{tikzpicture}
    \tikzexnodes
    \tikzexedges
  \end{tikzpicture}
\end{figure}

Une figure TikZ est constituée d'éléments définis à l'aide de commandes :

\begin{lstlisting}
  \commande[param£\`e£tres] ... suite de la commande ... ;
\end{lstlisting}

Par exemple, un graphe est composé de nœuds et d'arcs entre ces nœuds.
Tous sont définis à l'aide de commandes TikZ.

\medskip
On commence par définir une figure et une image TikZ :
\begin{lstlisting}
\begin{figure}
  \begin{tikzpicture}
    ...    % La figure ici
  \end{tikzpicture}
\end{figure}
\end{lstlisting}
\end{frame}

\begin{frame}[fragile, t]
  \frametitle{Exemple : un graphe simple}

\begin{figure}
  \begin{tikzpicture}
    \tikzexnodes
  \end{tikzpicture}
\end{figure}

On définit un nœud avec la commande \lstinline?\node? et on peut spécifier :
\medskip
\begin{itemize}
  \item le nom interne \lstinline?(nom)? et l'étiquette visible \lstinline?{etiquette}?
  \item la forme \quad \lstinline?circle?, \lstinline?ellipse?, \lstinline?square?, \lstinline?diamond?
  \item le type de ligne \quad \lstinline?draw? : visible, \lstinline?draw=blue? : en bleu, \lstinline?thick? : en gras
  \item la couleur de fond \quad \lstinline?fill=yellow? : fond jaune, \lstinline?shading? : dégradé
  \item la position \quad \lstinline?left/right/above/below of=n? : position par rapport à \lstinline?n?
\end{itemize}

\begin{lstlisting}
\node[circle, fill=yellow, draw]
  (rond) {1};
\node[ellipse, fill=red!50]
  (ellipse) [right of=1, node distance=3cm] {Une ellipse};
\node[diamond, fill=blue!50, draw=blue, thick]
  (diamantvide) [left of=1, node distance=2cm] {};
\end{lstlisting}
\end{frame}

\begin{frame}[fragile, t]
  \frametitle{Exemple : un graphe simple}

\begin{figure}
  \begin{tikzpicture}
    \tikzexnodes
    \tikzexedges
  \end{tikzpicture}
\end{figure}

On définit ensuite un arc entre deux nœuds avec la commande

\begin{lstlisting}
    \path[£\meta{options}£] (£\meta{origine}£) edge (£\meta{cible}£);
\end{lstlisting}

On peut définir le type de flèche (\lstinline?->?, \lstinline?o->?, \lstinline?-?),
la courbure (\lstinline?bend right?),
etc.

\begin{lstlisting}
\path[->] (rond) edge (ellipse);
\path[o->, bend right] (rond) edge (diamantvide);
\end{lstlisting}

On peut placer un nouveau nœud sur un arc avec le mot-clef \lstinline?node?.

\begin{lstlisting}
\path[-, bend right] (retour.east) edge (rond.south);
\path[o->>, bend right] (diamantvide) edge
  node[below, fill=green!30] (retour) {retour} (rond);
\end{lstlisting}
\end{frame}



\section{Conclusion sur TikZ}

\begin{frame}
  \frametitle{Réutiliser au maximum}

Pour produire de belles figures TikZ, le mieux est de chercher des exemples et de les modifier.

\begin{center}
Pour cela : \Huge Internet !
\end{center}

On pourra notamment se servir des exemples de TEXample.net \cite{tikzandpgfexamples}.

\bigskip
De plus, il est possible de :
\begin{itemize}
  \item définir des thèmes pour des figures semblables,
  \item d'utiliser des bibliothèques pour des diagrammes répandus (UML, schémas électriques...).
\end{itemize}
\end{frame}
