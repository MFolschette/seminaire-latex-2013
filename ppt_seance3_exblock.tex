\documentclass{beamer}

\usepackage[utf8]{inputenc}
\usepackage[T1]{fontenc}

\usepackage[french]{babel}

\usepackage{amsmath} 
\usepackage{graphicx}
\usepackage{wrapfig}
\usepackage{amsmath}
\usepackage{array}
\usepackage{amsfonts}
%\usepackage{mathtools}
\usepackage{amssymb}
\usepackage{bbm}
\usepackage{pdfpages} %inclusion de fichier pdf!
\usepackage[left,pagewise]{lineno}  
\usepackage{float}   %pour forcer les figuester ou elles sont !! 
\usepackage{multibib} %bibliomultiple  cf doc "multibib" sur le net
\usepackage[normalem]{ulem} %barrer un mot
\usepackage{fancyvrb}
\usepackage{listings, multicol, xcolor} % Listings
\usepackage{lmodern}
\usepackage{url, hyperref} % Urls

%pour colorer du \verb nom \colorverb
\usepackage{xcolor}



% \meta command, a la doc
\newcommand{\meta}[1]{\ensuremath\langle\itshape#1\ensuremath\rangle}

\lstset{
upquote=false,
columns=flexible,
basicstyle=\ttfamily\scriptsize,
language={[LaTeX]TeX},
identifiersstyle=\color{green},
emphstyle=\color{blue},
keywordstyle=\color{blue},
directivestyle=\color{blue},
commentstyle=\color{gray},
    inputencoding=utf8,
literate={eacute}{\'e}1,
    {eagrave}{\`e}1,
    {aagrave}{\`a}1,
    escapechar={£},
    moretexcs={meta},
morekeywords={
  part,chapter,subsection,subsubsection,
  frontmatter,mainmatter,backmatter,
  tableofcontents,listoffigures,listoftables,titlepage,
  includegraphics,includepdf,
  dddot,ddddot
}
}

\usetheme{Madrid}

\title{Séminaire \LaTeX, séance 3: utilisation avancée}
\date{jeudi 28 février 2013}
\author[Folschette, Jubien, Tanguy]{Maxime \textsc{Folschette\up{1} } \and Anthony \textsc{Jubien\up{2}} \and Julien \textsc{Tanguy\up{3}} \\ 
{\scriptsize  \up{1} IRCCyN équipe MeForBio \\
\up{2} IRCCyN équipe Robotique et ONERA Toulouse \\
\up{3} IRCCyN équipe Systèmes Temps Réels \\
maxime.folschette, anthony.jubien, julien.tanguy @irccyn.ec-nantes.fr} }
\institute[AED]{Association des Étudiants en Doctorat de l'ECN (AED)  \\ 
Document sous licence Creative Commons BY 3.0 FR \\
http://creativecommons.org/licenses/by/3.0/fr/}


\setbeamertemplate{navigation symbols}{}

\AtBeginPart{%
    \frame{\partpage}

    \begin{frame}
        \frametitle{Plan}
        \tableofcontents
    \end{frame}
}
 %author ect...

\begin{document}

%%%%%%%%% SLIDE %%%%%%%%%%%%%%%%%%

\begin{frame}[fragile]
  \frametitle{Exemple de thème : Madrid}

\begin{block}{Bloc normal (neutre)}
  Contenu du bloc (listes, équations, maths, ...)
\end{block}

\begin{alertblock}{Bloc d'alerte}
  Si on suppose :
  \begin{equation}
    1+1=0
  \end{equation}
  alors on peut prouver n'importe quoi.
\end{alertblock}

\begin{exampleblock}{Bloc d'exemple}
  Par exemple :
  \begin{itemize}
    \item Tout ce qui est vrai est aussi faux, et inversement,
    \item $x = y$ pour tout $x$ et tout $y$,
    \item mon chat et moi ne formons qu'une seule personne.
  \end{itemize}

\end{exampleblock}
\end{frame}


%%%%%%%%%%%%%%%%%%%%%%%%%%%%%%%%%%%%%%%%%%%%%%%%%%%%%%%%%%%%%%%%%%%%%%%%%%%%%%%%%%
%%%%%%%%%%%%%%%%%%%%%%%% FIN DU DOCUMENT %%%%%%%%%%%%%%%%%%%%%%%%%%%%%%%%%%%%%%%%%%%%%%%%
%%%%%%%%%%%%%%%%%%% ( NON PRISE EN COMPTE DE LA SUITE ) %%%%%%%%%%%%%%%%%%%%%%%%%%%%%%%%%%%%%%%%%%%
\end{document}
%%%%%%%%%%%%%%%%%%%%%%%%%%%%%%%%%%%%%%%%%%%%%%%%%%%%%%%%%%%%%%%%%%%%%%%%%%%%%%%%%%
%%%%%%%%%%%%%%%%%%%%%%%%%%%%%%%%%%%%%%%%%%%%%%%%%%%%%%%%%%%%%%%%%%%%%%%%%%%%%%%%%%
