\usepackage[utf8]{inputenc}
\usepackage[T1]{fontenc}

\usepackage[french]{babel}

\usepackage{amsmath} 
\usepackage{graphicx}
%\usepackage{wrapfig}
\usepackage{amsmath}
\usepackage{array}
\usepackage{amsfonts}
%\usepackage{mathtools}
\usepackage{amssymb}
\usepackage{bbm}
\usepackage{pdfpages} %inclusion de fichier pdf!
%\usepackage[left,pagewise]{lineno}  
%\usepackage{float}   %pour forcer les figuester ou elles sont !! 
\usepackage{multibib} %bibliomultiple  cf doc "multibib" sur le net
\usepackage[normalem]{ulem} %barrer un mot
\usepackage{fancyvrb}
\usepackage{listings, multicol, xcolor} % Listings
\usepackage{lmodern}
\usepackage{url, hyperref} % Urls

%pour colorer du \verb nom \colorverb
\usepackage{xcolor}



% \meta command, a la doc
\newcommand{\meta}[1]{\ensuremath\langle\itshape#1\ensuremath\rangle}

\lstset{
upquote=false,
columns=flexible,
basicstyle=\ttfamily\scriptsize,
language={[LaTeX]TeX},
identifiersstyle=\color{green},
emphstyle=\color{blue},
keywordstyle=\color{blue},
directivestyle=\color{blue},
commentstyle=\color{gray},
    inputencoding=utf8,
literate={eacute}{\'e}1,
    {eagrave}{\`e}1,
    {aagrave}{\`a}1,
    escapechar={£},
    moretexcs={meta},
morekeywords={
  part,chapter,subsection,subsubsection,
  frontmatter,mainmatter,backmatter,
  tableofcontents,listoffigures,listoftables,titlepage,
  includegraphics,includepdf,
  dddot,ddddot,
  frametitle,framesubtitle,
  pause,only,uncover,
  usetheme,usecolortheme,
  institute,maketitle,
  usetikzlibrary,
  node,path,
  commande
}
}

\usetheme{CambridgeUS}

\usepackage{tikz}
\usetikzlibrary{chains}
\usetikzlibrary{calc, decorations.pathmorphing, fadings, shadings, arrows, decorations.pathreplacing, shapes} 
